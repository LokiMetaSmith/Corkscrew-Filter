\documentclass[twocolumn]{article}
\usepackage{graphicx}
\usepackage{amsmath}
\usepackage{hyperref}
\usepackage[utf8]{inputenc}
\usepackage{geometry}
\geometry{a4paper, margin=1in}

\title{Autonomous Inverse Design of Helical Inertial Filters for Lunar Exploration: A Software-Defined Engineering Approach}
\author{Lawrence Kincheloe \\ \textit{NASA / Open Source Community}}
\date{\today}

\begin{document}

\maketitle

\begin{abstract}
As humanity prepares for sustained lunar presence, the mitigation of abrasive, electrostatic lunar regolith poses a critical challenge to Environmental Control and Life Support Systems (ECLSS). Traditional barrier filtration methods are prone to rapid clogging in such environments. This paper presents a novel "Software-Defined Engineering" framework for the autonomous design and optimization of helical inertial separators ("corkscrew filters"). By integrating parametric Computer-Aided Design (OpenSCAD), Computational Fluid Dynamics (OpenFOAM), and Generative Artificial Intelligence (LLM), the system performs an automated search of the high-dimensional design space. We demonstrate that this inverse design methodology can effectively navigate the complex fluid dynamic regime defined by Reynolds ($Re$), Dean ($De$), and Stokes ($Stk$) numbers. Experimental validation confirms the theoretical predictions, highlighting the critical dependence of separation efficiency on inlet velocity and the generation of stable Dean vortices.
\end{abstract}

\section{Introduction}

The Artemis program and the push for a permanent lunar outpost necessitate robust life support technologies capable of operating indefinitely with minimal maintenance. A primary failure mode for current ECLSS hardware is particulate loading. Lunar regolith, characterized by its jagged morphology and electrostatic adhesion, rapidly saturates standard High-Efficiency Particulate Air (HEPA) filters \cite{ref1}.

Inertial filtration offers a regenerative alternative. By utilizing centrifugal forces to separate particulates from the gas stream, these devices can theoretically operate without clogging. However, the design of such systems—specifically helical channel separators—involves complex trade-offs between separation efficiency and pressure drop ($\Delta P$). The fluid dynamics are governed by secondary flows (Dean vortices) that are highly sensitive to geometric parameters such as helix pitch, radius of curvature, and cross-sectional profile.

This paper explores a novel approach to solving this multi-objective optimization problem: \textit{Computer Automated Search Space Discovery}. Instead of manual iteration, we employ an autonomous agent that iteratively generates geometry, simulates performance, and "reasons" about the physics to propose design improvements.

\section{Methodology: The Autonomous Loop}

The design framework operates as a closed-loop feedback system, orchestrating three distinct software domains.

\subsection{Parametric Geometry Generation}
The physical domain is defined by a parametric model in OpenSCAD. Unlike static CAD files, this model is a function $f(x_1, x_2, ..., x_n) \rightarrow \text{STL}$, where inputs include:
\begin{itemize}
    \item \texttt{helix\_path\_radius}: Radius of the helical centerline.
    \item \texttt{helix\_profile\_scale\_ratio}: Elliptical distortion of the cross-section.
    \item \texttt{num\_bins}: Number of discrete trapping stages.
\end{itemize}
This "Configuration-as-Code" approach allows the optimization agent to manipulate geometry through text-based parameters.

\subsection{Computational Fluid Dynamics (CFD)}
The performance of each candidate design is evaluated using OpenFOAM v2406. The system employs the \texttt{simpleFoam} solver for steady-state, incompressible flow. A hexahedral-dominant mesh is generated using \texttt{snappyHexMesh} with specific boundary layer refinement to capture the wall shear stresses critical for accurate $\Delta P$ prediction.

\subsection{Agentic Reasoning}
The cognitive layer is driven by a Large Language Model (Google Gemini). The agent receives:
\begin{enumerate}
    \item Numerical metrics ($\Delta P$, separation efficiency).
    \item Multimodal feedback (images of the generated geometry).
    \item A history of prior iterations.
\end{enumerate}
Using "Chain-of-Thought" reasoning, the agent identifies trends in the high-dimensional state space (e.g., "increasing pitch reduced pressure drop but killed efficiency") and proposes the next set of parameters.

\section{Physics of Helical Filtration}

The separation mechanism relies on the generation of specific flow structures that emerge only within certain dimensionless parameter ranges.

\subsection{Reynolds Number ($Re$)}
The Reynolds number defines the turbulence regime:
\begin{equation}
Re = \frac{\rho U D}{\mu}
\end{equation}
Where $\rho$ is fluid density, $U$ is velocity, $D$ is hydraulic diameter, and $\mu$ is viscosity. High $Re$ is required to generate sufficient inertial forces, but excessive turbulence can cause re-entrainment of captured particles.

\subsection{Dean Number ($De$)}
In curved channels, centrifugal instability creates a pair of counter-rotating vortices known as Dean vortices. Their strength is governed by:
\begin{equation}
De = Re \sqrt{\frac{D}{2R_c}}
\end{equation}
Where $R_c$ is the radius of curvature. These vortices sweep the cross-section, transporting particles laterally. For effective separation, the Dean drag force must work in concert with the primary centrifugal force to move particles into the trapping zones ("gutters") on the outer wall.

\subsection{Stokes Number ($Stk$)}
The Stokes number characterizes the behavior of a particle suspended in a fluid stream:
\begin{equation}
Stk = \frac{\tau_p U}{D} = \frac{\rho_p d_p^2 U}{18 \mu D}
\end{equation}
\begin{itemize}
    \item $Stk \ll 1$: Particles follow fluid streamlines (no separation).
    \item $Stk > 1$: Particles detach from streamlines due to inertia (separation).
\end{itemize}

\section{Experimental Validation}

To validate the theoretical models, a series of physical prototypes were fabricated and tested using powdered sugar (mean diameter $\sim$10 $\mu$m) as a regolith proxy.

\subsection{Regime Comparison: 60 psi vs. 7.5 psi}
Tests were conducted at two distinct pressures to probe the sensitivity of the separation efficiency to the flow regime.

\textbf{Case 1: High Pressure (60 psi)}
\begin{itemize}
    \item Velocity: $\sim$40 m/s
    \item $Re$: $\approx$ 52,000 (Fully Turbulent)
    \item $De$: $\approx$ 32,000
    \item $Stk$: $\approx$ 1.02
\end{itemize}
\textit{Result:} The system achieved a separation ratio of 18:1. The high Dean number indicates aggressive secondary flows capable of sweeping particles to the walls, while $Stk > 1$ ensures inertial detachment.

\textbf{Case 2: Low Pressure (7.5 psi)}
\begin{itemize}
    \item Velocity: 5--10 m/s
    \item $Re$: 6,000 -- 13,000
    \item $De$: 4,000 -- 8,000
    \item $Stk$: 0.13 -- 0.26
\end{itemize}
\textit{Result:} The separation ratio dropped to 9:1. At this lower energy state, the particles behave more like gas ($Stk \ll 1$), following the streamlines and bypassing the traps. The Dean vortices, while present, lack the energy to overcome viscous drag.

\section{Search Space Discovery}

The experimental data reveals a non-linear "activation energy" for filtration: the system must exceed critical $Re$ and $De$ thresholds to function. A naive gradient-descent optimizer might fail to cross this threshold if initialized in a low-energy state.

The LLM-based agent, however, demonstrates the capacity to "jump" across the search space. By analyzing the correlation between \texttt{helix\_path\_radius} and predicted $\Delta P$, the agent learns to tighten the curvature ($R_c$) to boost $De$ without necessarily increasing inlet velocity, effectively navigating the trade-off curve.

\section{Conclusion}

The Corkscrew Filter project demonstrates that Software-Defined Engineering can accelerate the development of complex space hardware. By coupling parametric design with physics-based reasoning, we can explore the fluid dynamic state space more effectively than manual iteration. The theoretical analysis confirms that helical inertial filtration is a viable solution for lunar dust mitigation, provided the system is designed to operate above the critical Reynolds and Dean number thresholds identified in this study.

\begin{thebibliography}{9}
\bibitem{ref1}
J. R. Gaier, "The Effects of Lunar Dust on EVA Systems During the Apollo Missions," NASA Glenn Research Center, 2005.
\bibitem{ref2}
NASA Technology Transfer, "Corkscrew Filter Extracts Liquid From Air Charge," MSC-TOPS-118.
\end{thebibliography}

\end{document}
